\documentclass[12pt]{extarticle}

\usepackage{myStyle}

\title{Topology Project}
\author{Mostafa Touny}
\date{\today}

\begin{document}

\maketitle
\tableofcontents

\newpage

\section{Forward}
I had no time to prove \textit{Lemma 2}. The general \textit{Borsuk-Ulam} theorem requires advanced concepts in Algebraic Topology. The part of complex analysis and winding numbers is complete but needs more writing.


\section{Preliminaries}

\subsection{Graph Theory}

\textbf{Definition.} A coloring is proper if no adjacent vertices are assigned the same color.

\textbf{Definition.} The chromatic number $\chi(G)$ is the smallest number of colours to colour a graph.


\subsection{Complex Analysis}

\textbf{Definition.} A curve $\gamma$ in open $U \subseteq \mathbb{C}$ is
$$
\gamma: [a,b] \rightarrow U
$$

\textbf{Notation.} The curve $\gamma$ could be expressed as a parameteric function $f(e^{2 \pi i t})$ for $0 \leq t \leq 1$.

\textbf{Definition.} The continuity of a complex-valued function is defined in an analogous manner to real-valued functions.

\textbf{Definition.} For continuous $F:[a,b] \rightarrow \mathbb{C}$, the integral of $[a,b]$ is
$$
\int_a^b F(t) \; dt = \int_a^b u(t) \; dt + i \int_a^b v(t) \; dt
$$

\textbf{Definition.} The winding number with respect to point $\alpha$ over closed path $\gamma$ is
$$
W(\gamma,\alpha) = \frac{1}{2 \pi i} \int_\gamma \frac{1}{z - \alpha} \; dz
$$

\textbf{Definition.} The degree of a function $deg(f)$ is the wind number $W(f(e^{2 \pi it}))$.


\subsection{Other}

\textbf{Fact.} Vectors of d-dim sphere are exactly the vectors of the (d+1)-dim Euclidean space whose norm is $1$.

\textbf{Fact.} The equator of a d-dim sphere is a subspace of the d-dim Euclidean space.

\textbf{Definition.} The distance between a point $x$ and a closed set $C_i$ is $dist(x,C_i) = \inf_{y \in C_i} ||y-x||$, i.e the distance between $x$ and the closest point of $C_i$.

\textbf{Fact.} If point $p$ is in closed set $C_i$, then $dist(p, C_i) = 0$, and if not then $dist(p, C_i) > 0$.

\textbf{Definition.} Two points of a sphere are antipodal if they are diametrically opposite, i.e expressed as $p$ and $-p$.

\textbf{Definition.} The open hemisphere of pole $x$ is $H(x) = \{ y \in \mathbb{S}^d \mid \langle x,y \rangle > 0 \}$.


\section{Topological Methods}

\textbf{Lemma 1.} Given $f: S^1 \rightarrow S^1$, and $f(-z) = -f(z) \; \forall z \in S^1$, then $deg(f)$ is odd.

\textit{Proof.} Recall by definition $deg(f) = wind(f(e^{2 \pi it}))$. Set $g:[0,1] \rightarrow S^1$ by $g(t) = f(e^{2 \pi it})$.

If $z = e^{2\pi it}$ then $-z = e^{\pi i} = e^{\pi i}z = e^{\pi i} e^{2 \pi it} = e^{\pi i + 2 \pi it} = e^{2 \pi i (t + 1/2)}$.

If $0 \leq t \leq \frac{1}{2}$ then $\frac{1}{2} \leq t+\frac{1}{2} \leq 1$. We have $g(t+\frac{1}{2}) = f(e^{2 \pi i (t + 1/2)}) = f(-z) = -f(z) = -f(e^{2\pi it}) = -g(t)$.

For a partition of $[0,1]$ into $[0,\frac{1}{2}] \cup [\frac{1}{2}, 1]$, partition $[0, \frac{1}{2}]$ into $\bigcup_{k=1}^n [t_{k-1}, t_k]$, where $|\theta_k| < \frac{\pi}{2}$ for $\frac{g(t_k)}{g(t_{k-1})} = e^{i \theta_k}$. Partition $[\frac{1}{2}, 1] = \bigcup_{k=1}^n [\frac{1}{2} + t_{k-1}, \frac{1}{2} + t_k]$.

It follows $\frac{g(\frac{1}{2} + t_k)}{g(\frac{1}{2} + t_{k-1})} = \frac{-g(t_k)}{-g(t_{k-1})} = e^{i \theta_k}$.

Observe an approximation of the integral of the winding number would be
$$
\frac{1}{2\pi} \left ( \sum_{i=1}^n f(z_i) \cdot \Delta_i \right ) = \frac{1}{2\pi} (\theta_1 + \dots + \theta_n + \theta_1 + \dots + \theta_n)
$$

Observe $g(\frac{1}{2}) = -g(0)$, and
\begin{align*}
    \frac{g(t_n)}{g(t_{n-1})} \cdot \frac{g(t_{n-1})}{g(t_{n-2})} \cdot \dots \cdot \frac{g(t_2)}{g(t_1)} \cdot \frac{g(t_1)}{g(t_0)} &= \frac{t_n}{t_0} = \frac{1/2}{0} = -1 \\
    e^{i \theta_n} \cdot e^{i \theta_{n-1}} \cdot \dots \cdot e^{i \theta_2} \cdot e^{i \theta_1} &= \\
    e^{i(\theta_1 + \theta_2 + \dots + \theta_n)} &=
\end{align*}
But we know $e^{i \pi} = -1$. Hence $\theta_1 + \theta_2 + \dots + \theta_n = \pi + 2 \pi N$ for some integer $N$.

It follows the wind approximation is
$$
\frac{1}{2 \pi}((\pi + 2 \pi N) + (\pi + 2 \pi N)) = \frac{1}{2 \pi}(2 \pi + 4 \pi N) = 1 + 2N
$$
which is an odd number.

Indeed, a sequence of odd numbers converges to an odd number. Therefore, the winding number is odd.

\textbf{Lemma 2.} No map $f:S^2 \rightarrow S^1$ such that $f(-p) = -f(p) \; \forall p \in S^2$.

% \textit{Preliminary.} simply connected; equator homotopic to a point implying same degree.

% \textit{Sketch.} Assume the there exists such a map $f$ towards a contradiction. Take $g$ to be $f$ restricted to its equator. By \textit{Lemma 1}, the $deg(g)$ is odd. However, $S^2$ is simply connected, implying its equator can be shrunk or homotopic to a point. i.e $deg(g) = deg(\text{constant mapping}) = 0$. Contradiction.

\textbf{Theorem.} 2-dim Borsuk-Ulam. If $f: S^2 \rightarrow \mathbb{R}^2$ is continuous, then $\exists p \in S^2$ such that $f(-p) = f(p)$.

\textit{Proof.} Suppose towards contradiction $f(x) \neq f(-x) \; \forall x \in S^2$. Construct map $g: S^2 \rightarrow S^1$ by
$$
    g(x) = \frac{f(x) - f(-x)}{|| f(x) - f(-x) ||_2} \in S^1 \\
$$
Observe 
$$
    \forall x \in S^2 \quad g(-x) = \frac{f(-x) - f(-(-x))}{|| f(-x) - f(-(-x)) ||_2} = \frac{f(-x) - f(x)}{|| f(-x) - f(x) ||_2} = -g(x)
$$
Contradicting \textit{lemma 2}.

\textbf{Theorem.} Borsuk-Ulam. If $f: S^n \rightarrow \mathbb{R}^n$ is continuous, then $\exists p \in S^n$ such that $f(-p) = f(p)$.

The proof is a natural generalization but requires concepts beyond a first course in topology.

\textbf{Theorem.} Lyusternik \& Shnirel'man. If $S^n$ is covered by closed sets $C_1, C_2, \dots, C_n, C_{n+1}$, then there $p \in S^n$ and $C_i$ such that $p,-p \in C_i$.

\textit{Proof.} Assume towards contradiction that if $p \in C_i$ then $-p \not\in C_i$ for all $p \in S^n$. Define functions $f_1, f_2, \dots, f_{n+1}: S^n \rightarrow \mathbb{R}$ by $f_i(x) = dist(x, C_i)$. Construct $f: S^n \rightarrow \mathbb{R}^n$ by $f(x) = (f_1(x) - f_{n+1}(x), f_2(x) - f_{n+1}(x), \dots, f_n(x) - f_{n+1})$.

By \textit{Borsuk-Ulam} there are $p, -p \in S^n$ such that $f(p) = f(-p)$. It follows for all $1 \leq i,j \leq n+1$
\begin{align*}
    f_i(p) - f_{n+1}(p) &= f_i(-p) - f_{n+1}(-p)\\
    f_j(p) - f_{n+1}(p) &= f_j(-p) - f_{n+1}(-p)
\end{align*}
So $f_i(p) - f_j(p) = f_i(-p) - f_j(-p)$.

Since $p \in S^n = C_1 \cup \dots \cup C_{n+1}$, we get $p \in C_i$ for some $i$. Similarly $-p \in S^n$ so $-p \in C_j$ for some $j$. By hypothesis $i \neq j$. Clearly $f_i(p) = f_j(-p) = 0$. Then
\begin{align*}
    f_i(p) - f_j(p) &= f_i(-p) - f_j(-p) \\
    -f_j(p) &= f_i(-p)
\end{align*}
But $f_j(p) > 0$ since $p \not\in C_j$, and similarly $f_i(p) > 0$. So we have an equality between a negative and a positive number. Contradiction.


\section{Theorems}

\textbf{Definition.} The Kneser graph $KG_{n,k}$ for $n \geq 2$, $k \geq 1$, has vertex set $C([n], k)$, and any two vertices $u,v \in C([n], k)$ are adjacent if and only if they are disjoint, i.e. $u \cap v = \phi$.

\textbf{Theorem.} The chromatic number of the Kneser graph $KG_{n,k}$ is $n-2k+2$.

Fix $n$ and $k$. Assume for the sake of contradiction, the chromatic number of Kneser graph $KG_{n,k}$ is less than $n-2k+2$. Then we have a proper coloring $c: C([n], k) \rightarrow \{ 1, \dots, n-2k+1 \}$ using at most $n-2k+1$ colors. Set $d = n-2k+1$ and take a set $X$ of $n$ vectors on the d-dim sphere $\mathbb{S}^d$ where any $d+1$ vectors are linearly independent.

Let $U_i = \{ x \in \mathbb{S}^d \mid \exists k\text{-set} \; S \subset X, c(S) = i, S \subset H(x) \}$ for $i=1,\dots,d$, and take complement $A = \mathbb{S}^d \setminus (U_1 \cup \dots \cup U_d)$. We claim each $U_i$ is open.

To see why, fix a point $y \in S^{d}$, and observe $U_{y} = \{ x \in S^{n-1} : \langle x, y \rangle > 0 \}$ is open as it is the preimage of the open set $(0, \infty)$ under the continuous map $f_y(x) = \langle x, y \rangle$.\\

For finite k-subset $B = \{ y_1, \dots, y_k \}$, Observe
$$
U_B = \bigcap_{j=1}^k U_{y_j} = \left\{ x \in S^{n-1} : \langle x, y_j \rangle > 0 \ \forall j \right\}
$$
is an intersection of finitely many open sets, hence \textit{open}.

Therefore $U_i = \bigcup_{\substack{B \in \binom{[n]}{k} \\ c(B)=i}} U_B$ is a union of open sets, hence \textit{open}. Moreover complement $A$ is closed.

Clearly $A$ alongside $U_i$ do cover $\mathbb{S}^d$. So if none of them contains a pair of antipodal points, then neither does $\mathbb{S}^d$, hence contradicting the \textit{Lyusternik \& Shnirel'man} theorem. We aim to reach that contradiction. Consider $x \in \mathbb{S}^d$.

\textit{Case 1.} $x \in U_i$, i.e $H(x)$ contains a k-subset colored with color $i$, corresponding to a vertex colored $i$. Since $H(x)$ and $H(-x)$ are disjoint, any k-subset in $H(-x)$, is disjoint from any k-subset in $H(x)$. Thereby, corresponding vertices are adjacent. Since the coloring is proper by hypothesis, $H(-x)$ does not contain a k-subset colored with $i$, hence $-x \not\in U_i$.

\textit{Case 2.} $\pm x \in A$. By definition of $A$, neither $H(x)$ nor $H(-x)$ contains a k-subset of $X$. Hence each of $H(x)$ and $H(-x)$ contains at most $k-1$ vectors. It follows there is at least $n-2(k-1) = n-2k+2 = d+1$ points in the equator $\{ y \in \mathbb{S}^d \mid \langle x,y \rangle = 0 \}$, contained in a subspace of dim d, concluding they are linearly dependent. Contradiction.

Finally we show a valid constructive coloring of $KG_{n,k}$ using $n-2k + 2$ colors. Color each k-set with all elements in $[2k - 1]$ with one color, and every other k-set by their largest element. Thereby we use at most $n - (2k-1) + 1 = n - 2k + 2$ colors, where all k-sets of a given color intersect.

\section{Resources}

\begin{itemize}
    \item \href{https://www.youtube.com/watch?v=gk6Unu78e-w}{Borsuk-Ulam Theorem by Nicolas Bourbaki}
    \item \href{https://www.youtube.com/watch?v=zG9Y1ksp3Qw}{Kneser's Conjecture by Nicolas Bourbaki}
    \item \href{https://link.springer.com/book/10.1007/978-1-4419-7910-0}{A Course in Topological Combinatorics, Chapter 2, by Longueville}
    \item \href{http://math.mit.edu/~fox/MAT307-lecture14.pdf}{Lecture 14. Combinatorics by Jacob Fox}
\end{itemize}

\end{document}
